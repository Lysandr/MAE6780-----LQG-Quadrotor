\documentclass[conf]{new-aiaa}
%\documentclass[journal]{new-aiaa} for journal papers
\usepackage[utf8]{inputenc}

\usepackage{graphicx}
\usepackage{amsmath}
\usepackage[version=4]{mhchem}
\usepackage{siunitx}
\usepackage{longtable,tabularx}
\setlength\LTleft{0pt} 

\title{A Convex Optimal and LQG Control Approach for Quadcopters}

\author{Pádraig Lysandrou\footnote{Undergraduate Student, Electrical and Computer Engineering, AIAA Student Member}} \affil{Cornell University, Ithaca, New York, 14853}


\begin{document}

\maketitle

\begin{abstract}
The independent project report shall be a maximum of 15 pages in double-spaced format,
including all graphs and data tables, not including title page or reference section. The report shall
include the sections listed below.

The abstract shall be one paragraph, giving a brief description of the motivation and purpose of the project and also listing key results.\end{abstract}

\section*{Nomenclature}

{\renewcommand\arraystretch{1.0}
\noindent\begin{longtable*}{@{}l @{\quad=\quad} l@{}}
$A$  & amplitude of oscillation \\
$a$ &    cylinder diameter \\
$C_p$& pressure coefficient \\
$Cx$ & force coefficient in the \textit{x} direction \\
$Cy$ & force coefficient in the \textit{y} direction \\
c   & chord \\
d$t$ & time step \\
$Fx$ & $X$ component of the resultant pressure force acting on the vehicle \\
$Fy$ & $Y$ component of the resultant pressure force acting on the vehicle \\
$f, g$   & generic functions \\
$h$  & height \\
$i$  & time index during navigation \\
$j$  & waypoint index \\
$K$  & trailing-edge (TE) nondimensional angular deflection rate
\end{longtable*}}


\clearpage
\begin{doublespace}

\section{Introduction and Theory}
This section shall introduce the \textbf{multivariate} control approach and describe relevant theory background in approximately three to five page(s). The Theory section is one of the most important sections of the paper, as this relays to the reader your knowledge of the subject matter.

\section{Procedure}
This section describes the steps involved in the control design and tuning. It also shall describe
the simulation setup and include explanatory diagrams and figures when appropriate. As a rule of
thumb, this should be written with the intention of giving step-by-step instructions to someone
else to complete the same design and be able to obtain your exact result. This shall be done in
approximately three to five pages.

\section{RESULTS}

This section should be the meat of your report. It should be three to six pages. If the control
design and validation have multiple components, you may use sub-sections as necessary. Include
a description of simulation findings, using diagrams, figures, and numerical data measurements
(by means of tables) that provide a convincing argument on the effectiveness and shortcomings of
your chosen multivariable controller, in the context of the original design objectives. It may be
appropriate to put some of the raw data into Appendices. Plot the results in a meaningful way- “a
plot is worth a thousand words,” do NOT underestimate the power of a good plot! Consider
using statistical analysis if appropriate when completing a plot. Verify your findings through the
theory. Possible causes of error should also be addressed in the concluding paragraph.


\section{Conclusions}
Explain in one paragraph what are the main contributions of your project, the main theoretical
findings, and provide a summary of the control performance along with any recommendations for
improvements through future work.

\section{ref}
Be sure to cite any references you used in this lab report using approved IEEE standards. 





\end{doublespace}
\bibliography{sample}
\end{document}
